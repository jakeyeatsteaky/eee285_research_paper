\begin{titlepage}
    \begin{center}
        \vspace*{-20mm}
        \huge
        \textbf{Data Flow Architecture:\\ \vspace{-3mm} Benefits, Limitations and Future Applications}\\
         \large
        
        \vspace{0.3 cm}
        \textbf{
        Jake Baker (jacobwilliambaker@csus.edu) \\
        \vspace{-2mm}
        Jesse Harkness (jesseharkness@csus.edu)} \\ 
        \vspace{-2mm}
        EEE285 - Micro-Computer System Design I \\
        \vspace{-2mm}
        INSERT/ACTUAL/DATE\\
        \vspace{10mm}
        \includegraphics[width=0.3\textwidth]{Picture Files/1200px-California_State_University,_Sacramento_seal.svg.png}
        \vfill
        \textbf{Abstract}\\
        \raggedright\setlength{\parindent}{3em}
        The automation of everyday tasks is the magnum opus of computer and information technologies; as computer processing ability continues to increase, the next ubiquitous technological advance will be through artificial intelligence (AI).  But the level of computation required for AI applications comes at a cost; increasing transistor density is no longer a feasible method for increasing compute power.  In contrast to the traditional CPU architecture which relies on smaller and denser spreads of transistors, Data Flow Architecture (DFA) is a hardware design which maximizes compute potential by reducing data movement, and increasing work efficiency through highly specialized processors.  By exploiting the availability of data, rather than the sequence of instructions, DFA improves concurrency of operations and has the potential to maximize throughput.  Comparing the similarities and differences of the dataflow paradigm with a traditional CPU highlights the benefits of DFA, some of the inherent challenges, as well as the future possibilties for this hardware architecture in the AI space.     
\par


    \end{center}
\end{titlepage}